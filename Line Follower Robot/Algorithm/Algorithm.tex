\documentclass[journal,12pt,twocolumn]{IEEEtran}

\usepackage{setspace}
\usepackage{gensymb}
\usepackage{hyperref}
\hypersetup{
    colorlinks=true,
    linkcolor=blue,
    filecolor=magenta,      
    urlcolor=cyan,
    pdftitle={Overleaf Example},
    pdfpagemode=FullScreen,
    }

\urlstyle{same}
\singlespacing


\usepackage[cmex10]{amsmath}

\usepackage{amsthm}

\usepackage{mathrsfs}
\usepackage{txfonts}
\usepackage{stfloats}
\usepackage{bm}
\usepackage{cite}
\usepackage{cases}
\usepackage{subfig}

\usepackage{longtable}
\usepackage{multirow}

\usepackage{enumitem}
\usepackage{mathtools}
\usepackage{steinmetz}
\usepackage{tikz}
\usepackage{circuitikz}
\usepackage{verbatim}
\usepackage{tfrupee}
\usepackage[breaklinks=true]{hyperref}
\usepackage{graphicx}
\usepackage{tkz-euclide}

\usetikzlibrary{calc,math}
\usepackage{listings}
    \usepackage{color}                                            %%
    \usepackage{array}                                            %%
    \usepackage{longtable}                                        %%
    \usepackage{calc}                                             %%
    \usepackage{multirow}                                         %%
    \usepackage{hhline}                                           %%
    \usepackage{ifthen}                                           %%
    \usepackage{lscape}     
\usepackage{multicol}
\usepackage{chngcntr}

\DeclareMathOperator*{\Res}{Res}

\renewcommand\thesection{\arabic{section}}
\renewcommand\thesubsection{\thesection.\arabic{subsection}}
\renewcommand\thesubsubsection{\thesubsection.\arabic{subsubsection}}

\renewcommand\thesectiondis{\arabic{section}}
\renewcommand\thesubsectiondis{\thesectiondis.\arabic{subsection}}
\renewcommand\thesubsubsectiondis{\thesubsectiondis.\arabic{subsubsection}}


\hyphenation{op-tical net-works semi-conduc-tor}
\def\inputGnumericTable{}                                 %%

\lstset{
%language=C,
frame=single, 
breaklines=true,
columns=fullflexible
}
\begin{document}


\newtheorem{theorem}{Theorem}[section]
\newtheorem{problem}{Problem}
\newtheorem{proposition}{Proposition}[section]
\newtheorem{lemma}{Lemma}[section]
\newtheorem{corollary}[theorem]{Corollary}
\newtheorem{example}{Example}[section]
\newtheorem{definition}[problem]{Definition}

\newcommand{\BEQA}{\begin{eqnarray}}
\newcommand{\EEQA}{\end{eqnarray}}
\newcommand{\define}{\stackrel{\triangle}{=}}
\bibliographystyle{IEEEtran}
\providecommand{\mbf}{\mathbf}
\providecommand{\pr}[1]{\ensuremath{\Pr\left(#1\right)}}
\providecommand{\qfunc}[1]{\ensuremath{Q\left(#1\right)}}
\providecommand{\sbrak}[1]{\ensuremath{{}\left[#1\right]}}
\providecommand{\lsbrak}[1]{\ensuremath{{}\left[#1\right.}}
\providecommand{\rsbrak}[1]{\ensuremath{{}\left.#1\right]}}
\providecommand{\brak}[1]{\ensuremath{\left(#1\right)}}
\providecommand{\lbrak}[1]{\ensuremath{\left(#1\right.}}
\providecommand{\rbrak}[1]{\ensuremath{\left.#1\right)}}
\providecommand{\cbrak}[1]{\ensuremath{\left\{#1\right\}}}
\providecommand{\lcbrak}[1]{\ensuremath{\left\{#1\right.}}
\providecommand{\rcbrak}[1]{\ensuremath{\left.#1\right\}}}


\theoremstyle{remark}
\newtheorem{rem}{Remark}
\newcommand{\sgn}{\mathop{\mathrm{sgn}}}
\providecommand{\abs}[1]{\left\vert#1\right\vert}
\providecommand{\res}[1]{\Res\displaylimits_{#1}} 
\providecommand{\norm}[1]{\left\lVert#1\right\rVert}
%\providecommand{\norm}[1]{\lVert#1\rVert}
\providecommand{\mtx}[1]{\mathbf{#1}}
\providecommand{\mean}[1]{E\left[ #1 \right]}
\providecommand{\fourier}{\overset{\mathcal{F}}{ \rightleftharpoons}}
%\providecommand{\hilbert}{\overset{\mathcal{H}}{ \rightleftharpoons}}
\providecommand{\system}{\overset{\mathcal{H}}{ \longleftrightarrow}}
	%\newcommand{\solution}[2]{\textbf{Solution:}{#1}}
\newcommand{\solution}{\noindent \textbf{Solution: }}
\newcommand{\cosec}{\,\text{cosec}\,}
\providecommand{\dec}[2]{\ensuremath{\overset{#1}{\underset{#2}{\gtrless}}}}
\newcommand{\myvec}[1]{\ensuremath{\begin{pmatrix}#1\end{pmatrix}}}
\newcommand{\mydet}[1]{\ensuremath{\begin{vmatrix}#1\end{vmatrix}}}
\numberwithin{equation}{subsection}
\makeatletter
\@addtoreset{figure}{problem}
\makeatother
\let\StandardTheFigure\thefigure
\let\vec\mathbf
\renewcommand{\thefigure}{\theproblem}
\def\putbox#1#2#3{\makebox[0in][l]{\makebox[#1][l]{}\raisebox{\baselineskip}[0in][0in]{\raisebox{#2}[0in][0in]{#3}}}}
     \def\rightbox#1{\makebox[0in][r]{#1}}
     \def\centbox#1{\makebox[0in]{#1}}
     \def\topbox#1{\raisebox{-\baselineskip}[0in][0in]{#1}}
     \def\midbox#1{\raisebox{-0.5\baselineskip}[0in][0in]{#1}}
\vspace{3cm}
\title{Arduino Based Line Follower Robot}
\author{K.A. Raja Babu}
\maketitle
\newpage
\bigskip
\renewcommand{\thefigure}{\theenumi}
\renewcommand{\thetable}{\theenumi}
\tableofcontents

\section{ALGORITHM}
\subsection{\textbf{Components}}
\url{https://github.com/ka-raja-babu/Arduino-Based-Robot/blob/main/Line%20Follower%20Robot/Component%20%20List.pdf}
\subsection{\textbf{Wiring Diagram}}
\url{https://github.com/ka-raja-babu/Arduino-Based-Robot/blob/main/Line%20Follower%20Robot/Wiring%20Diagram.pdf}

\numberwithin{table}{section}
\begin{table}[!ht]
\centering
\begin{tabular}{|c|c|} 
\hline
Motor Shield & Left IR Sensor  \\
\hline
D7 & D \\ 
\hline
GND & GND \\ 
\hline
5V & 5V \\ 
\hline
\end{tabular}
\caption{Connection for Left IR Sensor}
\label{tab:table1}
\end{table}

\numberwithin{table}{section}
\begin{table}[!ht]
\centering
\begin{tabular}{|c|c|} 
\hline
Motor Shield & Right IR Sensor  \\
\hline
D8 & D \\ 
\hline
GND & GND \\ 
\hline
5V & 5V \\ 
\hline
\end{tabular}
\caption{Connection for Right IR Sensor}
\label{tab:table1}
\end{table}

\subsection{\textbf{Arduino Code}}
\begin{itemize}
    \item Connect the Arduino uno board to Laptop/PC using USB cable.
    \item Open the \href{https://github.com/ka-raja-babu/Arduino-Based-Robot/blob/main/Line%20Follower%20Robot/Code.ino}{Code.ino} file in Arduino IDE. 
    \item From Tools menu, select Board as "Arduino Uno" and suitable "Port" on which the Arduino board is connected.
    \item Compile the code by clicking on "Verify" option.
    \item Upload the code to Arduino Uno using the "Upload" option.
\end{itemize}
\subsection{\textbf{Working}}
\begin{enumerate}
    \item Working of a line follower robot is based on the IR transmitters and receivers which are also called photo diodes .
    \item When infrared rays transmitted by IR transmitter, fall on a white surface,rays are reflected back and received by IR receiver which generates some voltage changes. 
    \item When infrared rays transmitted by IR transmitter, fall on a black surface,rays are absorb by the black surface and no rays are reflected back, thus IR receiver does not receive any light and no voltage changes occur.
    \item When IR sensor senses white surface, then arduino gets 1 as input and when sensor senses black surface, arduino gets 0 as input.
    \item IR sensor can be calibrated according to need of the user.
\end{enumerate}
\newpage
\subsection{\textbf{Images}}

\numberwithin{figure}{section}
\begin{figure}[!ht]
\centering
\includegraphics[width=\columnwidth]{Line Follower Robot 1.jpeg}
\caption{Line Follower Robot}
\end{figure}

\numberwithin{figure}{section}
\begin{figure}[!ht]
\centering
\includegraphics[width=\columnwidth]{Line Follower Robot 2.jpeg}
\caption{Path of Line Follower Robot}
\end{figure}

\end{document}
